%%%%%%%%%%%%%%%%%%%%%%%%%%%%%%%%%%%%%%%%%
% Medium Length Professional CV
% LaTeX Template
%
% Original author:
% Trey Hunner (http://www.treyhunner.com/)
%
% Important note:
% This template requires the resume.cls file to be in the same directory as the
% .tex file. The resume.cls file provides the resume style used for structuring the
% document.
%
%%%%%%%%%%%%%%%%%%%%%%%%%%%%%%%%%%%%%%%%%

%----------------------------------------------------------------------------------------
%   PACKAGES AND OTHER DOCUMENT CONFIGURATIONS
%----------------------------------------------------------------------------------------

\documentclass{resume} % Use the custom resume.cls style

\usepackage[left=0.75in,top=0.6in,right=0.75in,bottom=0.6in]{geometry} % Document margins

\name{Ales Huzik} % Your name
% \address{Minsk, Belarus} % Your address
% \address{+375~29~575-1103 \\ me@aguzik.net} % Your phone number and email
\address{Sydney, Australia} % Your address
\address{+61~427~990-909 \\ me@aguzik.net} % Your phone number and email

\begin{document}

%----------------------------------------------------------------------------------------
%   EDUCATION SECTION
%----------------------------------------------------------------------------------------

\begin{rSection}{Education}

{\bf Belarusian State University of Informatics and Radioelectronics, Minsk} \hfill {\em July 2013} \\ 
Faculty of Computer Systems and Networks \\
M.S. in Computer Science \& Software Engineering \smallskip \\
{\bf Belarusian State University of Informatics and Radioelectronics, Minsk} \hfill {\em July 2012} \\ 
Faculty of Computer Systems and Networks \\
B.S. in Systems Engineering \\

\end{rSection}

%----------------------------------------------------------------------------------------
%   TECHNOLOGY SECTION
%----------------------------------------------------------------------------------------

\begin{rSection}{Have experience with}

\begin{tabular}{ @{} >{\bfseries}l @{\hspace{6ex}} l }
Operating Systems     & Linux (NixOS, Arch, Debian, CentOS, Ubuntu), Mac OS X \smallskip \\
Programming Languages & Clojure, Ruby, C, Bash, JavaScript, Perl, Common Lisp, Erlang, \\
                      & Maude, Factor, Lua, Terra, C++, Python and some others\smallskip \\
Markup and Typesetting & \LaTeX, HTML, Haml, Slim, CSS, SASS/SCSS, Less, Bootstrap \smallskip \\
Server technologies   & NixOps, Consul, Docker, Dokku, Chef, Nginx, Apache httpd, \\
                      & Postfix, Prosody, vsftpd, Squid, Corosync, Pacemaker, DRBD \smallskip \\
SQL Databases & PostgreSQL, MySQL/MariaDB, SQLite \smallskip \\
  NoSQL Databases & Datomic/Datascript, Neo4j, OrientDB, MongoDB, Cassandra \smallskip \\
% Tools & zsh, git, emacs, vim, tmux, gpicker, ctags, the silver searcher, \\
%       & GNU Global, splint, gdb, strace, valgrind, gprof \smallskip \\
Programming paradigms & Imperative, Object-oriented (class-based, prototype-based), \\
                      & Functional, Concatenative (stack-based), Logical (predicate logic, \\
                      & rewriting logic).
\end{tabular}

\end{rSection}

%----------------------------------------------------------------------------------------
%   WORK EXPERIENCE SECTION
%----------------------------------------------------------------------------------------

\begin{rSection}{Work positions}

\begin{rSubsection}{Atlassian}{December 2016 - Present}{Senior Clojure Developer}{Sydney, Australia}

% At the moment I have worked only 7.5 months. It is somewhat hard to evaluate my own performance, as I always think that I could have done so much more. Also, because I often tend to be interested in big, architecture-level changes, they go slower than I'd wish and it's harder to show. Nevertheless, since I joined the team I have done the following things:
% - maintainability improvements:
%   - reorganization of the code base
%     - decluttered repository toplevel
%     - cleaned up obsolete/abandoned branches (with automatic script to do that periodically)
%   - shrinked application bundle size twice (from 100mb to 50mb), which made deployments noticeably faster
%   - formulated automated versioning strategy in a way that does not require conscious effort from the developers, and is helpful for the server team to understand amount and severity of changes.
%   - proposed and added jvm memory consumption metrics, so we can see when GC happens and can understand it's implications on the dynamic behavior of the system
% - productivity
%   - automation of manual deployments. allows developers to test changes in ddev/adev/staging faster. instead of commiting, and clicking your way through bamboo you can deploy your local code tree directly with just one command.
%   - automation of versioning. Developers don't need to review all code changes since last version change and manually edit the version. After a while you don't even notice that you are doing versioning.
% - Readability and understandability of the code
%   - Insisted on starting doing proper code reviews. Found a compromise to start doing that after going 100% GA
%   - created a glossary of terms for increasing consistency and noticing ambiguities in naming we use, so that we can improve it.
%   - have been doing code reviews. gave suggestions on improving naming
%   - insisted on strict adherence to our code style guide
% - Architecture
%   - Reworking build process (cleaned up a lot of obsolete stuff, simplified clojurescript and uberjar build process, working on splitting synchrony into smaller subprojects)
%   - Dockerization
%   - Grand vision of splitting synchrony into smaller, easier to maintain pieces
%     - part 1: clustering
%       - extracted into synchrony-clustering microservice
%       - synchrony part is decoupled from the mechanism of getting list of active nodes, so hazelcast-based (or even completely static) implementation can be created for ConfServer synchrony, keeping synchrony identical between cloud and server. 
%     - part 2: accepting client connections and routing them to correct synchrony processing nodes
%       - have been accepted as a valuable option by Tobias
%       - prerequisites for the implementation have been tested
%       - turned out impossible to implement due sticky sessions being unsupported for websockets in AWS, and direct connections to nodes being disallowed by micros
%       - investigated an option to use Kubernetes (at that moment it turned out to not be ready for deploying production services)
%       - have recently found a way to overcome the limitations of aws/micros, so desired architecture could be implemented in a relatively straightforward way. will pursue it after finishing rework of the build pipeline 
%   - Led Synchrony Architecture Bootcamp (02/05/2017).
%   - Participated in architecture review (05/06/2017).
% - Got into Ship-It finals
% Other than that I have participated in interviewing potential candidates, have been helping with onboarding, have been pairing with other team members, doing on-call job, and working with support to investigate customer issues (sometimes during after hours). 
% Regarding company values - most of the stuff above is BTCYS, which is the value that resonates the most with me. Help with onboarding, pairing, substituting other team members for on-call time is PAAT. Working with support investigating customer issues, and regular work to make Synchrony more stable/reliable is DFTC. I think that calling up on issues with the codebase and Tobias could be labeled as OCNB, though I'm still unsure if my actions were right.
~
\end{rSubsection}

\begin{rSubsection}{Filemporium}{July 2015 - December 2016}{Lead Clojure Developer}{Remote via Upwork}
\item Setup temporary deplyoment via docker and dokku
\item Revised architecture in a way that drastically simplified client-side
  state management and allowed live page update of all active user sessions
\item Reengineered project build system using boot (previously leiningen were used).
  Fixed project build time (full recompilation now takes just a couple of minutes
  instead of an hour). Adjusted project code to work with reloaded workflow.
\item Documented project structure, project-specific code conventions, technical decisions, troubleshooting, and Amazon S3 project-specific step-by-step configuration guide
\item Implemented production-ready multiserver setup with zero-downtime deployment using NixOS/NixOps and Consul
\item Refactored most of the project, implemented lots of functionality and fixed lots of bugs (e.g added
  config schema validation, cleaned up garbage logging (like {\tt (println "!!! FOO:" x)}) and implemented
  propper configurable logging throughout the system, implemented chunked file upload with automatic reconnection, etc.)
\item Interviewed potential candidates
\item Regularly did code reviews
\item Did pair programming (to assist others with complicated tasks, to share project knowledge, to get
  back on track when I'm stuck)
\item Fired a programmer that have been writing terrible code
\end{rSubsection}

\begin{rSubsection}{Filemporium}{May 2015 - July 2015}{Clojure Developer}{Remote via Upwork}
\item Automated design updates
\item Added compile-time template checks to kioo templating library to ensure
  component correctness after design update
\item Added support for using arbitrary npm libraries from ClojureScript code (to be
  able to utilize existing js React components)
\item Started writing project documentation. Documented actions needed to setup
  a project, update design, add npm library

% ~
\end{rSubsection}

\begin{rSubsection}{Softswiss Casino Software}{October 2014 - May 2015}{Senior Software Engineer}{Minsk, Belarus}
\item Implemented integrations with external game providers (CasinoTechnology, Fengaming)
\item Implemented completely custom design for new customer (HTML/CSS)
\item Worked on external wallet api implementation
\end{rSubsection}

\begin{rSubsection}{Rubyroid Labs, LLC}{April 2014 - September 2014}{Senior Software Engineer/Team Leader}{Minsk, Belarus}
\item Designed application architecture
\item Managed project development
\item Did code reviews
\item Solely implemented some internal services
\end{rSubsection}

\begin{rSubsection}{Intetics Co.}{July 2013 - April 2014}{Senior Software Engineer}{Minsk, Belarus}
\item Made fully-automated production server setup
\item Worked on refactoring legacy codebase
\item Worked on security-related features (IP whitelisting, XSS testing)
\item Implemented backend service for mobile apps.
\item Implemented automatic management of VPN servers DNS rotation
\item Did code reviews
\end{rSubsection}

\begin{rSubsection}{\parbox[t][2em][t]{9cm}{Belarusian State University of Informatics and Radioelectronics}}{February 2013 - January 2014 }{Teaching assistant at Electronic Computing Machines Department (part-time)}{Minsk, Belarus}
\item Taught first-year students programming in C.
\item Taught fourth-year students IP networking.
\item Taught students how to use Git and GitHub.
\item Together with students formalized grading criteria.
\item Formalized some code quality metrics.
\item Regularly reviewed students' code.
\item Taught Linux for interested students in my spare time.
\end{rSubsection}

\begin{rSubsection}{PowerMeMobile, Inc.}{January 2013 - February 2013}{Problem solver}{Minsk, Belarus}
\item Gave an idea of automating deployment process (new tier deployments may take up to
  a month of SysAdmin team work).
\item Implemented initial stages of deployment automation (installing base cluster software,
  configuring corosync/pacemaker, installing and configuring DRBD and nginx as resource agents) using Chef.
\item Made entire deployment configurable from a single place (from chef workstation using node attributes).
\item Got an agreement on opensourcing this efforts.
\end{rSubsection}

\begin{rSubsection}{Altoros Systems, Inc.}{October 2011 - September 2012}{Software Engineer in Ruby department}{Minsk, Belarus}
\item Proved that custom multisite functionality is a bad idea. Dropped the hacks and refactored application to use rails 3 engines.
\item Participated in porting internal RightScale services (mostly sinatra+cassandra) to JRuby to utilize native Thrift.
\item Participated in all stages of design and development on many projects.
\end{rSubsection}

\begin{rSubsection}{Itransition, Inc.}{February 2011 - October 2011}{Junior Developer in Ruby department}{Minsk, Belarus}
\item Solely ported large social gaming engine from Rails 2 to Rails 3.
\item Initiated using SCSS and Compass, which led to stylesheets development and modification speedup.
\item Configured production server from scratch and setup automated Capistrano deployment.
\end{rSubsection}

\end{rSection}

%----------------------------------------------------------------------------------------
%   NOTES SECTION
%----------------------------------------------------------------------------------------

\begin{rSection}{Some facts to better understand what kind of person I am}
  \smallskip
  \begin{list}{$\cdot$}{\leftmargin=0em} % \cdot used for bullets, no indentation
    \itemsep -0.5em \vspace{-0.5em} % Compress items in list together for aesthetics
  \item I decided to tie my work to computers when I was 5.
  \item First program in BASIC at age of 11, first HTML and JavaScript at 12, first program in Pascal at 13.
  \item I started playing with Linux when I was 14 (it was Mandrake 10 in 2005)
  \item I use Dvorak keyboard layout
  \item I use Archlinux with lots of handwritten scripts and i3 tiling window manager as my work environment.
  \item I tend to automate everything I could.
  \item I started using Emacs at 2010 and never looked back. Since 2011 I use it with Evil (vim emulation layer)
  \item I have dozens of personal opensource projects and have contributed to
        upstream of at least 20 other.
  \item Some time ago I was passionate about japanese animation, so I learned
        some japanese and passed an international exam (JLPT4 certificate). I remember myself
        learning Esperanto and Toki Pona, and now I am learning Lojban.
  \item I am an active ACM and ACM SIGPLAN member.
  \item One of my primary interests is programming languages. I am still thinking of one
        that would be perfectly expressive. I have plans for PhD in that field.
  \item I write tests very rarely and when I feel I wish to write one, I know that
    I should search for more simple solution that will be obviously working and not
    require any iterative validation. I do not believe that writing tons
    of tests may lead to good system design. More often it leads to design that is hard to change
    in any meaningful way. Instead of going test-first I go think-first, and don't start writing
    code until I clearly know what I'm going to write and why.
  \end{list}
\end{rSection}

%----------------------------------------------------------------------------------------
%   INTERESTS SECTION
%----------------------------------------------------------------------------------------

\begin{rSection}{Growth directions}
  \smallskip
  \begin{list}{$\cdot$}{\leftmargin=0em} % \cdot used for bullets, no indentation
    \itemsep -0.5em \vspace{-0.5em} % Compress items in list together for aesthetics
  \item Programming paradigms and programming languages
  \item Rewriting logic (Maude system in particular)
  \item Minimal syntax programming languages: Lisp (Clojure, Common Lisp, Scheme,
        Qi/Shen), FORTH, Factor, APL, Tcl, Refal, Rebol, Smalltalk
  \item Programming music (Overtone, Extempore) and visuals (Quil, Processing, Fluxus)
  \item Learning to draw
  \item Semantics
  \item Constructed languages
  \item Neuroscience
  \item Statistics and Machine Learning
  \end{list}
\end{rSection}

%----------------------------------------------------------------------------------------

\end{document}
